\begin{frame}
Morphisms that are associated with each other according to the bijections of an adjunction are called {\it adjoint transposes} of one another. 

There is a correspondence
\begin{block}{}
\abovedisplayskip=0pt
\begin{align*}
g &: Fc \rightarrow d, \,\, g \in \Mor(\mathcal{D})\\
g^* &: c \rightarrow Gd, \,\, g^* \in \Mor(\mathcal{C})
\end{align*}
\end{block}
given by $\phi_{c,d}(g) = g^*$. 
Similarly for 
\begin{block}{}
\abovedisplayskip=0pt
\begin{align*}
f &: c \rightarrow Gd, \,\, f \in \Mor(\mathcal{C})\\
f^* &: Fc \rightarrow d, \,\, f^* \in \Mor(\mathcal{D}) 
\end{align*}
\end{block}
given by $\phi_{c,d}^{-1}(f) = f^*$. 
We see then that $g^* = f$ and $f^* = g$.
\end{frame}