\begin{frame}
We can then give an alternative definition of adjoint functors in terms of the counit natural transformation (dually the unit natural transformation) as
$$
F \colon \mathcal{C} \rightleftarrows \mathcal{D} \colon G
$$
$$
\epsilon \colon FG \rightarrow 1_{\mathcal{C}}
$$
where for any $c \in \Ob (\mathcal{C})$, $d \in \Ob (\mathcal{D})$, and $g \colon Fc \rightarrow d \in \Mor(\mathcal{D})$ there exists a unique $f \colon c \rightarrow Gd \in \Mor(\mathcal{C})$ such that $g = \epsilon_D \circ Ff$	
\begin{columns}[t]
    \begin{column}{0.5\textwidth}
\begin{block}{counit conditions for $F \dashv G$}
$$
			\xymatrix{
			& F c \ar[d]^{F f} \ar[dl]_{g} & c \ar@{.>}[d]^{f}\\
			d & F G d \ar[l]^{\epsilon_d} & G d}
$$
\end{block}
    \end{column}
    \begin{column}{0.5\textwidth}
		\begin{block}{adjoint correspondence}
		\abovedisplayskip=0pt
		$$
			\frac{Fc \longrightarrow d}{c \longrightarrow Gd}
		$$
		\end{block}
    \end{column}
\end{columns}
\end{frame}