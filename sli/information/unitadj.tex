\begin{frame}[t]
We can then give an alternative definition of adjoint functors in terms of the unit natural transformation (dually the counit natural transformation) as
\begin{align*}
F & \colon \mathcal{C} \rightleftarrows \mathcal{D} \colon G\\
\eta & \colon 1_{\mathcal{C}} \rightarrow GF
\end{align*}
where for any $c \in \Ob (\mathcal{C})$, $d \in \Ob (\mathcal{D})$, and $f \colon c \rightarrow Gd \in \Mor(\mathcal{C})$ there exists a unique $g \colon Fc \rightarrow d \in \Mor(\mathcal{D})$ such that $f = Gg \circ \eta_c$				
\begin{columns}[t]
    \begin{column}{0.5\textwidth}
     \begin{block}{unit conditions for $F \dashv G$}
		\abovedisplayskip=0pt
		$$
					\xymatrix{
					c \ar[r]^{\eta_c} \ar[dr]_{f} & G F c \ar[d]^{G g} & F c \ar@{.>}[d]^{g}\\
					& G d & d}
		$$
		\end{block}
    \end{column}
    \begin{column}{0.5\textwidth}
		     \begin{block}{adjoint correspondence}
		\abovedisplayskip=0pt
		$$
			\frac{c \longrightarrow Gd}{Fc \longrightarrow d}
		$$
		\end{block}
    \end{column}
\end{columns}
\end{frame}