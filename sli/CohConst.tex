\subsection{left adjoint}
\begin{frame}
This section relies on and follows the notation of the proof of Giraud's theorem. Given a functor $A: \mathcal{L}_0 \rightarrow \mathcal{L}_1$, we can recall the adjunction $L_A \dashv R_A$ between the category of presheaves on the local category $\mathcal{L}_0$ and the global category $\mathcal{L}_1$:
$$
L_A: \textit{Sets}^{\mathcal{L}_0^{opp}} \leftrightarrows \mathcal{L}_1 :R_A.
$$
\end{frame}


\begin{frame}
We can describe the adjoint pair parameterized by a presheaf $P \in \textit{Sets}^{\mathcal{L}_0^{opp}}$, an object $L_1 \in \Ob(\mathcal{L}_1)$ and and object $L_0 \in \Ob(\mathcal{L}_0)$ as
\begin{eqnarray*}
L_A(P) &=& \lim\limits_{\longrightarrow} \left( \int P \xrightarrow{\pi_P} \mathcal{L}_0 \xrightarrow{A} \mathcal{L}_1 \right),\\
R_A(L_1)(L_0) &=& Mor_{\mathcal{L}_1}(A(L_0),L_1).
\end{eqnarray*}
\end{frame}

\begin{frame}
The functor $L_A$ can also be constructed as a coequalizer
\begin{displaymath}
\xymatrix{
\coprod\limits_{\substack{u : L'_0 \rightarrow L_0 \\ p \in P(L_0)}}
A(L'_0)
\ar@<1ex>[r]^-{\theta} \ar@<-1ex>[r]_-{\tau}
&
\coprod\limits_{\substack{L_0 \\ p \in P(L_0)}}
A(L_0)
\ar[r]
&
P(L_0) \otimes_{\mathcal{L}_0} A
}
\end{displaymath}
where $P(L_0) \otimes_{\mathcal{L}_0} A = L_A(P(L_0))$.
\end{frame}

\begin{frame}
For a suitable topology $J$ on $\mathcal{L}_0$, the functor $R_A = \Mor_{\mathcal{L}_1}(A(-),-)$ sends $L_1 \in \Ob(\mathcal{L}_1)$ into sheaves (as opposed to mere presheaves) on $\mathcal{L}_0$. $J$ is defined by specifying that a sieve $S(L_0)$ on an object $L_0 \in \mathcal{C}$ is a cover of $L_0$ when the morphisms $g_i : M_0^i \rightarrow L_0 \in S(L_0)$ form an epimorphic family in $\mathcal{L}_1$. The morphisms $g_i$ are components of a morphism defined on the coproduct
$$
b_{S(L_0)} : \coprod\limits_{g_i \in S(L_0)} M_0^i \longrightarrow L_0
$$
over $S(L_0)$ . $S(L_0)$ is defined to cover $L_0$ when $b_{S(L_0)}$ is an epimorphism in $\mathcal{L}_1$, which will be demonstrated in more detail using $A:\mathcal{L}_0 \rightarrow \mathcal{L}_1$.
Since this construction is parametric over $L_0 \in \Ob(\mathcal{L}_0)$ and satisfies the maximality, stability and transitivity conditions, this defines a Grothendieck topology on $\mathcal{L}_0$. 
\end{frame}


\begin{frame}
The conditions of maximality, stability under pullback, and transitivity that ensure the sieve is a Grothendieck topology can all be checked. Roughly, the maximality condition is satisfied by including $1_{L_0} : L_0 \rightarrow L_0$ in the set of morphisms $g_i$ for some $i$. The stability condition requires more detail to verify. Consider the pullback square of an epimorphic family along a morphism $h:L'_0 \rightarrow L_0$
\begin{displaymath}
\xymatrix{
\coprod\limits_{g_i \in S(L_0)} M_0^i \times_{L_0} L'_0 \ar[r]^-{\tilde{g_i}} \ar[d]_{} & L'_0 \ar[d]^{h} \\
\coprod\limits_{g_i \in S(L_0)} M_0^i \ar[r]^-{g_i} & L_0
}
\end{displaymath}
where $\tilde{g_i}$ is also an epimorphism. 
\end{frame}

\begin{frame}
\begin{enumerate}
\item For $\{ g_i : M_0^i \rightarrow L_0 \}_i \in S(L_0)$ there is an epimorphic family of morphisms $\{ N_0^{ij} \rightarrow M_0^i \times_{L_0} L'_0 \}_{ij}$ with each $N_0^{ij} \in \Ob(\mathcal{L}_0)$. 
\pause \item Therefore the composites of these morphisms, $N_0^{ij} \rightarrow M_0^i \times_{L_0} L'_0 \rightarrow L'_0$, for all $i$ and $j$ form an epimorphic family on $L'_0$ contained in the sieve given by the preimage of $h$ on the sieve $S(L_0)$, $h^*(S(L_0))$, and thus $h^*(S(L_0))$ is a cover on $L'_0$, which satisfies the definition of stability of $S(L_0)$ under pullback.
\pause \item Transitivity can also be verified, and thus $J$ is shown to be a Grothendieck topology on $\mathcal{L}_0$ defined in terms of a condition---that sieves on $\mathcal{L}_0$ cover objects in $\mathcal{L}_1$---on its relationship to $\mathcal{L}_1$.
\end{enumerate}
\end{frame}

\subsection{right adjoint}
\begin{frame}
Next we explain the way in which the right adjoint functor parametrically applied to an object $L_1 \in \Ob(\mathcal{L}_1)$, $R_A(L_1) = \Mor_{\mathcal{L}_1}(A(-),L_1): \mathcal{L}_0^{opp} \rightarrow Sets$, is interpreted as a sheaf for the topology $J$ on $\mathcal{L}_0$.
\end{frame}

\begin{frame}
Assuming that $\mathcal{L}_1$ is an exact category, the epimorphic family $b_{S(L_0)}$ is the coequalizer of the pullback of $b_{S(L_0)}$ along itself. Since coproducts are stable under pullback for the given assumptions on $\mathcal{L}_1$, 
$$
\left( \coprod\limits_{g'_i \in S(L_0)} M_{0}^{'i} \right) \times_{L_0} \left( \coprod\limits_{g_i \in S(L_0)} M_0^i \right) \cong \coprod\limits_{g'_i , g_i \in S(L_0)} M_0^{'i} \times_{L_0} M_0^i
$$
so that the coequalizer takes the form
\begin{displaymath}
\xymatrix{
\coprod\limits_{g'_i , g_i \in S(L_0)} M_0^{'i} \times_{L_0} M_0^i
\ar@<1ex>[r]^-{\sigma_1} \ar@<-1ex>[r]_-{\sigma_2}
&
\coprod\limits_{g_i \in S(L_0)} M_0^i
\ar[r]^-{b_{S(L_0)}}
&
L_0
}
\end{displaymath}
\end{frame}

\begin{frame}
Due to transitivity of $S(L_0)$ there is an epimorphic family represented by $\{ N_0^{ij} \rightarrow M_0^{'i} \times_{L_0} M_0^i \}_{ij}$ that we can precompose with the coequalizer above yielding
\begin{displaymath}
\xymatrix{
\coprod\limits_{i,j} N_0^{ij}
\ar@<1ex>[r]^-{\alpha} \ar@<-1ex>[r]_-{\beta}
&
\coprod\limits_{g_i \in S(L_0)} M_0^i
\ar[r]^-{b_{S(L_0)}}
&
L_0
}
\end{displaymath}
where there is a commutative diagram in $\mathcal{L}_0$ of the following form
\begin{displaymath}
\xymatrix{
N_0^{ij} \ar[r]^-{h} \ar[d]_{k} & M_0^i \ar[d]^{g_i} \\
M_0^{'i} \ar[r]_-{g'_i} & L_0
}
\end{displaymath}
for each $L_0 \in \Ob(\mathcal{L}_0)$ and where $\alpha$ is associated to $h$ and $\beta$ is associated to $k$.
\end{frame}

\begin{frame}
Now we can apply the functor $R(L_1): \mathcal{L}_0^{opp} \rightarrow Sets$ to the previous coequalizer to obtain
\begin{displaymath}
\xymatrix{
R(L_1)(L_0) \ar[r]
&
\prod\limits_{g_i \in S(L_0)} R(L_1)(M_0^i)
\ar@<1ex>[r]^-{} \ar@<-1ex>[r]_-{}
&
\prod\limits_{i,j} R(L_1)(N_0^{ij})
}
\end{displaymath}
where the arrows are reversed and coproducts exchanged with products producing an equalizer due to the contravariance of the functor $R(L_1)$. 
\end{frame}

\begin{frame}
\begin{enumerate}
\item This equalizer provides precisely the condition necessary for $R(L_1)$ to satisfy the sheaf condition for the covering sieve $S(L_0)$. 
\item Since this construction is parametric in $L_0 \in \Ob(\mathcal{L}_0)$ for a corresponding $L_1 \in \Ob(\mathcal{L}_1)$ of the topology $J$ then $R$ is a $J$-sheaf for the site $(\mathcal{L}_0,J)$. 
\item For those cases in which $L_1$ represents an object in $\Ob(\mathcal{L}_0)$ then $R(L_1)$ is equivalent to the functor represented by the object $L_0$ in the presheaf category on $\mathcal{L}_0$. 
\item When all representable functors are sheaves, the topology $J$ is referred to as the subcanonical topology on $\mathcal{L}_0$.
\end{enumerate}
\end{frame}

\subsection{unit and counit}
\begin{frame}
Now we consider the unit and counit natural transformations of the adjunction $L_A \dashv R_A$ in the context of the functor $A: \mathcal{L}_0 \rightarrow \mathcal{L}_1$ that enables the transformation of information between the local and the global levels of organization represented by the respective categories.
\end{frame}

\begin{frame}
For each $L_1 \in \Ob(\mathcal{L}_1)$, the counit natural transformation
\begin{eqnarray*}
\epsilon_{L_1} &:& L \circ R(L_1) \rightarrow L_1\\
&:& R(L_1) \otimes_{\mathcal{L}_0} A \rightarrow L_1\\
&:& \Mor_{\mathcal{L}_1}(A(-),L_1) \otimes_{\mathcal{L}_0} A \rightarrow L_1
\end{eqnarray*}
can be defined in the coequalizer diagram
\begin{displaymath}
\xymatrix{
\coprod\limits_{\substack{u : L'_0 \rightarrow L_0 \\ r:A(L_0) \rightarrow L_1}}
A(L'_0)
\ar@<1ex>[r]^-{\theta} \ar@<-1ex>[r]_-{\tau}
&
\coprod\limits_{\substack{L_0 \in \Ob(\mathcal{L}_0) \\ r:A(L_0) \rightarrow L_1}}
A(L_0)
\ar[r]^-{\pi} \ar[dr]_-{b_{A(L_0),r}}
&
L_A R_A(L_1) \ar[d]^-{\epsilon_{L_1}}\\
& & L_1
}
\end{displaymath}
where the functor $R_A$ sends objects $L_1 \in \Ob(\mathcal{L}_1)$ to $J$-sheaves on $\mathcal{L}_0$. 
\end{frame}

\begin{frame}
The morphisms $\theta$ and $\tau$ are associated to the red and blue paths through the diagrams
\begin{displaymath}
\xymatrix{
& A(L_0) \ar@[red][d]^{r_i} \\
A(L_0^{'}) \ar@[red][ur]^{A(u)} \ar@[blue][r]_-{r'_i} & L_1
}
\end{displaymath}
\end{frame}

\begin{frame}
The counit $\epsilon_{L_1}$, parametric in $L_1$ is in fact an isomorphism. This can be seen by considering the epimorphic family
$$
b_{A(L_0),r} : \coprod\limits_{\substack{L_0 \in \Ob(\mathcal{L}_0) \\ r:A(L_0) \rightarrow L_1}}
A(L_0) \longrightarrow L_1
$$
in $\mathcal{L}_1$ defined in terms of the functor $A$ and objects $L_0$. This epimorphic family is part of a coequalizer diagram
\begin{displaymath}
\xymatrix{
\coprod\limits_{\substack{i,j}}
A(N_0^{ij})
\ar@<1ex>[r]^-{\alpha} \ar@<-1ex>[r]_-{\beta}
&
\coprod\limits_{\substack{r_i:A(M_0^i) \rightarrow L_1}}
A(M_0^i)
\ar[r]^-{b_{A(M_0^i),r_i}} \ar[dr]_-{\pi}
&
L_1 \ar[d]^-{\epsilon_{L_1}^*}\\
& & L_A R_A(L_1)
}
\end{displaymath}
\end{frame}

\begin{frame}
where $i,j$ indexes commutative diagrams in $\mathcal{L}_1$ of the form
\begin{displaymath}
\xymatrix{
A(N_0^{ij}) \ar@[red][r]^-{h} \ar@[blue][d]_{k} & A(M_0^i) \ar@[red][d]^{r_i} \\
A(M_0^{'i}) \ar@[blue][r]_-{r'_i} & L_1
}
\end{displaymath}

%The morphisms $\alpha$ and $\beta$ are respectively associated to the red and blue paths through the diagrams. In this case, $b$ is a coequalizer for $\alpha$ and $\beta$ if and only if it is also a coequalizer for $\theta$ and $\tau$.
\end{frame}

\begin{frame}
This fact is evident in the argument \footnote{We allow the substitution rules $L_0 \rightarrow M_0^i$ and $L'_0 \rightarrow M_0^{'i}$ since these differences are merely notational in the sense that they only serve to index objects in the category $\mathcal{L}_0$.} that $(b \theta = b \tau) \Rightarrow (b \alpha = b \beta)$
\begin{eqnarray*}
(b \theta = b \tau) &\Rightarrow & b_{A(M_0^i),r_i} \circ h \\
&=& b_{N_0^{ij},r_i h}\\
&=& b_{N_0^{ij},r'_i k}\\
&=& b_{A(M_0^{'i}),r'_i} \circ k\\
&\Rightarrow & (b \alpha = b \beta)
\end{eqnarray*}
\end{frame}

\begin{frame}
and the converse $(b \alpha = b \beta) \Rightarrow (b \theta = b \tau)$
\begin{eqnarray*}
(b \alpha = b \beta) &\Rightarrow & b_{A(M_0^i),r_i} \circ A(u) \\
&=& b_{A(M_0^{'i}),r_i A(u)}\\
&\Rightarrow & (b \theta = b \tau).
\end{eqnarray*}
This demonstrates that $\epsilon_{L_1}$ is indeed an isomorphism and thus
\begin{eqnarray*}
\epsilon_{L_1} &:& L \circ R(L_1) \cong L_1\\
\epsilon &:& L \circ R \cong 1_{\mathcal{L}_1}
\end{eqnarray*}
\end{frame}

\begin{frame}
We can also demonstrate that the unit of the adjunction is an isomorphism. For any presheaf $Q \in Sets^{\mathcal{L}_0^{opp}}$ the unit of the adjunction $L_A \dashv R_A$ is 

\begin{eqnarray*}
\eta_{Q} &:& Q \rightarrow R \circ L(Q)\\
&:& Q \rightarrow \Mor_{\mathcal{L}_1}(A(-),L_A(Q))\\ 
&:& Q \rightarrow \Mor_{\mathcal{L}_1}(A(-), Q \otimes_{\mathcal{L}_0} A)
\end{eqnarray*}
\end{frame}

\subsection{equivalence of categories}
\begin{frame}
For the case in which $Q$ is also a sheaf on $\mathcal{L}_0$, it can be dually characterized in terms of a colimit of representable functors $PSh(L_0) = Mor_{\mathcal{L}_0}(-,L_0)$. Additionally, the representable functors are sheaves for the case of the subcanonical topology $J$ on $\mathcal{L}_0$ defined above. 
\end{frame}

\begin{frame}
Thus if we consider the component of the unit natural transformation at any representable functor we find an isomorphism
$$
\eta_{PSh(L_0)}: PSh(L_0) \cong \Mor_{\mathcal{L}_1}(A(-), PSh(L_0) \otimes_{\mathcal{L}_0} A)
$$
because $PSh(L_0) \cong \Mor_{\mathcal{L}_1}(A(-), A(L_0)$ and $A(L_0) \cong L_A(PSh(L_0))$ in the category $\mathcal{L}_1$. Because the functors $L_A$ and $R_A$ both preserve colimits, any sheaf $Q$ is a colimit of representables, and $\eta_{PSh(L_0)}$ is an isomorphism for representable functors, then $\eta$ is an isomorphism for any sheaf $Q$.
\end{frame}

\begin{frame}
We have demonstrated the counit, $\epsilon$, and the unit $\eta$, to both be isomorphisms for the adjunction $L_A \dashv R_A$ upon restriction from $Sets^{\mathcal{L}_0^{opp}}$ to sheaves on the site defined by $J$ in terms of epimorphic families of morphisms in $\mathcal{L}_1$ for $\mathcal{L}_0)$ written as $Sh(\mathcal{L}_0,J)$ 
\begin{eqnarray*}
\eta &:& 1_{Sh(\mathcal{L}_0,J)} \cong R_A \circ L_A,\\
\epsilon &:& L_A \circ R_A \cong 1_{\mathcal{L}_1}.
\end{eqnarray*}
\end{frame}

\begin{frame}
We thus have an equivalence of categories between the category of local representations of biological information and that of global representations
$$
Sh(\mathcal{L}_0,J) \cong \mathcal{L}_1.
$$
\end{frame}

\begin{frame}
\begin{enumerate}
\item In this light, we can view a process in which interactions among local entities generate higher-level properties that themselves also become entities that may participate in still higher-level processes as requiring the type of coherence precisely specified by the sheaf condition. 
\item As biological systems are comprised of a nested hierarchy, this constraint, as applied to models of particular biological systems in a category, not limited to but including the graphs which are often produced from network data, can be used to link models across levels of organization.
\end{enumerate}
\end{frame}