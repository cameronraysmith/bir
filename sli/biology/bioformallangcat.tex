\begin{frame}
\begin{block}{}
Category theory \cite{Lane1985,Lane1998,MacLane1992,Lawvere1997,Lawvere2003,Awodey2006} is a language that has been suggested since Woodger to provide a framework for representing and reasoning about biological systems \cite{Rashevsky1954,Rosen1958,Rosen1958a,Rosen1978,GOGUEN1979,Rosen1985,Rosen1991,Ehresmann2007,Louie2009} 
\end{block}
\begin{block}{}
What is immediately useful about this language from the perspective of biology and its underlying chemistry is that it presents as primitive the notion of transformation between objects. 
\end{block}
\begin{block}{}
From this point of view, a defining characteristic of any representation of an entity (e.g. a protein, cell, organism, or population) is the set of relationships between its representations of other entities under consideration. 
\end{block}
\begin{block}{}
Intrinsic properties are taken into account implicitly in this framework since the possible set of relationships between any object and any other is constrained by the nature of its intrinsic properties. 
\end{block}
\end{frame}
