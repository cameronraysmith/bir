\begin{frame}
\begin{block}{}
The \textbf{language} we use to express these constraints is sufficiently general that it applies equally to several different concrete formalisms for representing biological systems.
\end{block}
\begin{block}{}
For example, \textbf{algebraic structures} that have become commonplace in the context of \textbf{network science}: algebraic graphs, hypergraphs, or (cell) complexes. 
\end{block}
\begin{block}{}
The \textbf{derived relationship} between levels of organization in models of biological systems can be used to \textbf{constrain dynamical models} defined in terms of \textbf{transformations on graphs}, or another concrete relational structure, that seek to account for hierarchical transitions in evolutionary processes.
\end{block}
\end{frame}