\begin{frame}
For a suitable topology $J$ on $\mathcal{L}_0$, the functor $R_A = \Mor_{\mathcal{L}_1}(A(-),-)$ sends $L_1 \in \Ob(\mathcal{L}_1)$ into sheaves (as opposed to mere presheaves) on $\mathcal{L}_0$. $J$ is defined by specifying that a sieve $S(L_0)$ on an object $L_0 \in \mathcal{C}$ is a cover of $L_0$ when the morphisms $g_i : M_0^i \rightarrow L_0 \in S(L_0)$ form an epimorphic family in $\mathcal{L}_1$. The morphisms $g_i$ are components of a morphism defined on the coproduct
$$
b_{S(L_0)} : \coprod\limits_{g_i \in S(L_0)} M_0^i \longrightarrow L_0
$$
over $S(L_0)$ . $S(L_0)$ is defined to cover $L_0$ when $b_{S(L_0)}$ is an epimorphism in $\mathcal{L}_1$, which will be demonstrated in more detail using $A:\mathcal{L}_0 \rightarrow \mathcal{L}_1$.
Since this construction is parametric over $L_0 \in \Ob(\mathcal{L}_0)$ and satisfies the maximality, stability and transitivity conditions, this defines a Grothendieck topology on $\mathcal{L}_0$. 
\end{frame}