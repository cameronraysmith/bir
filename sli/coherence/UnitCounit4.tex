\begin{frame}
The counit $\epsilon_{L_1}$, parametric in $L_1$ is in fact an isomorphism. This can be seen by considering the epimorphic family
\begin{block}{}
$$
b_{A(L_0),r} : \coprod\limits_{\substack{L_0 \in \Ob(\mathcal{L}_0) \\ r:A(L_0) \rightarrow L_1}}
A(L_0) \longrightarrow L_1
$$
\end{block}
in $\mathcal{L}_1$ defined in terms of the functor $A$ and objects $L_0$. This epimorphic family is part of a coequalizer diagram
\begin{block}{}
\begin{displaymath}
\xymatrix{
\coprod\limits_{\substack{i,j}}
A(N_0^{ij})
\ar@<1ex>[r]^-{\alpha} \ar@<-1ex>[r]_-{\beta}
&
\coprod\limits_{\substack{r_i:A(M_0^i) \rightarrow L_1}}
A(M_0^i)
\ar[r]^-{b_{A(M_0^i),r_i}} \ar[dr]_-{\pi}
&
L_1 \ar[d]^-{\epsilon_{L_1}^*}\\
& & L_A R_A(L_1)
}
\end{displaymath}
\end{block}
\end{frame}