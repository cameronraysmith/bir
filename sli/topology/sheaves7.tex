\begin{frame}
\begin{block}{encoding a category of {\it higher-order information structures} in terms of {\it lower-order information structures}}
Given these different views on the sheaf concept, we have the necessary concepts to explain how covering sieves on $\mathcal{L}_0$ defined in terms of epimorphic families of morphisms in $\mathcal{L}_1$ provides a Grothendieck topology on $\mathcal{L}_0$, which induces a site and sheaves on $\mathcal{L}_0$. This can be shown to precisely encode the category $\mathcal{L}_1$.
\end{block}
\begin{block}{}
\begin{displaymath}
%\xymatrix{
%& \mathcal{L}_1 \ar@<0.5ex>[rd]^-{R} & \\
%\mathcal{L}_0 \ar[ru]^{A} \ar[rr]_{PSh} & & \textit{Sets}^{\mathcal{L}_0^{opp}} \ar@<0.5ex>[lu]^-{L}
%}
\xymatrix{
& \mathcal{L}_1 \ar@<0.5ex>[rd]^-{R} \ar@<0.5ex>[rr]^{R} & & Sh(\mathcal{L}_0,J) \ar@<0.5ex>[ll]^{L} \ar@{^{(}->}@<0.5ex>[dl]^{} \\
\mathcal{L}_0 \ar[ru]^{A} \ar@{^{(}->}[rr]_{PSh} & & \hat{\mathcal{L}_0} \ar@<0.5ex>[lu]^-{L} \ar@<0.5ex>[ur]^{}&
}
\end{displaymath}
\end{block}
\end{frame}