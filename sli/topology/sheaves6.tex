\begin{frame}
According to the {\it matching family} definition, a presheaf $P \in \textit{Sets}^{\cC^{opp}}$ is a sheaf for the topology $J$ generated by the basis $B$ if and only if for any cover $\{ f_i : c_i \rightarrow c \}_{i \in I}$ in $B$ any matching family $\{ x_i \}_i$ has a unique amalgamation $x \in P(c)$ such that $x \cdot f_i = x_i$ for all $i \in I$. 
\begin{block}{definition of sheaf in terms of a {\it basis}}
This condition can also be expressed in terms of a diagram that is an equalizer
\begin{displaymath}
\xymatrix{
P(c) \ar[r]^-{e}
&
\prod\limits_{i \in I}
P(c_i)
\ar@<1ex>[r]^-{p_1} \ar@<-1ex>[r]_-{p_2}
&
\prod\limits_{i,j \in I \times I} P(c_i \times_c c_j)
}
\end{displaymath}
\noindent The morphism 

$e$ refers to $e(x) = \{ x \cdot f_i \}_i$, 

$p_1$ refers to $p_1(\{ x_i \})_{i,j} = x_i \cdot \pi_{ij}^1$, and 

$p_2$ refers to $p_2(\{x_i\})_{i,j} = x_j \cdot \pi_{ij}^{2}$.
\end{block}
\end{frame}