\begin{frame}
\begin{columns}
\begin{column}{0.5\textwidth}
\begin{block}{cones and cocones}
Combining the diagrams for limit and colimit (in general the objects W refer to different objects in each diagram) demonstrates the reason that limits are often referred to as {\it cones over} a diagram and colimits are referred to as {\it cocones under} a diagram
\end{block}
\end{column}
\begin{column}{0.5\textwidth}
\begin{block}{}
\abovedisplayskip=0pt
\begin{displaymath}
\xymatrix@=17pt{
&\ar[ldd]_{q_i} W \ar@{-->}[d]^{q} \ar[rdd]^{q_{i'}} & \\
& \ar[ld]^{p_i} \lim\limits_{\longleftarrow \mathcal{I}} M \ar[rd]_{p_{i'}} &\\
M_i  \ar[rr]_{M(\phi)} & & M_{i'} \\
i \ar[rr]^{\phi} & & i' \\
M_i \ar[rd]^{s_i} \ar[rdd]_{t_i} \ar[rr]^{M(\phi)} & & M_{i'} \ar[ldd]^{t_{i'}} \ar[ld]_{s_{i'}}\\
& \lim\limits_{\longrightarrow \mathcal{I}} M \ar@{-->}[d]^{t}& \\
& W &
}
\end{displaymath}
\end{block}
\end{column}
\end{columns}
\end{frame}