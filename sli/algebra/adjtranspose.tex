\begin{frame}
\begin{block}{adjoint transpose}
Morphisms that are associated with each other according to the bijections of an adjunction are called {\it adjoint transposes} of one another. If $g:Fc \rightarrow d$, $g \in \Mor(\mcD)$ then $g^*: c \rightarrow Gd$, $g^* \in \Mor(\cC)$ is given by $\phi_{c,d}(g) = g^*$. Similarly for $f: c \rightarrow Gd$, $f \in \Mor(\cC)$ with $f^*:Fc \rightarrow d$, $f^* \in \Mor(\mcD)$ is given by $\phi_{c,d}^{-1}(f) = f^*$. We see then that $g^* = f$ and $f^* = g$.
\end{block}
\end{frame}
%Morphisms that are associated with each other according to the bijections of an adjunction are called {\it adjoint transposes} of one another. If $g:u(X) \rightarrow Y$, $g \in \Mor(\mcD)$ then $g^*: X \rightarrow v(Y)$, $g^* \in \Mor(\cC)$ is given by $\phi_{X,Y}(g) = g^*$. Similarly for $f: X \rightarrow v(Y)$, $f \in \Mor(\cC)$ with $f^*:u(X) \rightarrow Y$, $f^* \in \Mor(\mcD)$ is given by $\phi_{X,Y}^{-1}(f) = f^*$. We see then that $g^* = f$ and $f^* = g$.