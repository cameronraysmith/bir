%-----------------------------------------------%
%             filename: skeleton.tex
%-----------------------------------------------%
\documentclass[aps,twocolumn]{revtex4-1}
\usepackage{graphicx}
\usepackage{ifpdf}
\ifpdf
	\usepackage[backref]{hyperref}
	%\usepackage[backref,pageanchor=true,plainpages=false, pdfpagelabels,bookmarks,bookmarksnumbered]	{hyperref}
\else
\fi

\usepackage{crs}

% see http://goo.gl/5Fo27
\newtoggle{thmsty}
%\toggletrue{thmsty}
\togglefalse{thmsty}

% see http://goo.gl/7jLZ9
\makeatletter
\newlength \figwidth
\if@twocolumn
  \setlength \figwidth {0.8\columnwidth}
\else
  \setlength \figwidth {0.5\textwidth}
\fi
\makeatother

\begin{document} 

\title{\bf A language for biology}

\author{a1$^{1}$, a2$^{1}$, a3$^{1}$, a4$^{1,2,3}$}

\affiliation{$^1$Department of Systems and Computational Biology,\\ $^2$Dominick P. Purpura Department of Neuroscience, \\ $^3$Department of Pathology, Albert Einstein College of Medicine, 1301 Morris Park Ave, Bronx, NY 10461, USA}

\date{\today}
\begin{abstract}
One of the most important aspects of advancing theory in biology is the development of a constructed language that can be explicitly and precisely defined thereby supporting communication, computer-based knowledge representation, and hypothesis-generating thought. Such a language should itself be evolvable and support flexible abstractions that enable the unification and compression of redundant information via the expression of principles of invariance. It would be useful for such a language to be formally specified prior to or in concert with the development of a computational implementation that may support complementary organization of existing and future data. We propose the specialization of a language originally developed in a mathematical context that may be capable of meeting these criteria. We argue for this choice, not from a mathematical perspective but, by providing an example of the way in which this language enables precise qualitative reasoning and flexible methods of abstraction with respect to integrating biological information in a manner that would, ideally, be analogous to the way in which information is integrated by biological systems themselves. Enabling compression and representation of synthetic knowledge about biological systems, in addition to raw biological data, is necessary for understanding in the context of limited data and computational resources.
\end{abstract}

\maketitle

\tableofcontents

\section{Introduction}

\subsection{Domain-specific language construction for biology}

\begin{quotation}
{\it One [...] idea, which underlies everything in this book, is the concept
of genomically encoded information processing. [...],
this is like the geological basis of the landscape. In my view, cis-regulatory information processing, and information processing at the gene regulatory network circuit level, are the real secret of animal development. Probably the appearance of genomic regulatory systems capable of information processing is what made animal evolution possible.} -Eric H. Davidson \cite{Davidson2006a}
\end{quotation}

The development of a formal language for modeling and reasoning about biological systems was suggested by Joseph Woodger in collaboration with the developmental biologist Conrad Waddington and the logician Alfred Tarski as early as 1937 \cite{Woodger1937,Woodger1951,Woodger1952,Woodger1952a}. At that time it was perhaps difficult to understand how such a language could be put to use. Today we have tools that could enable the use of such a language: namely 1) computational machines to automate the details of routine transformations within the language and 2) large, accessible, growing repositories of biological data. Though we have access to necessary infrastructure, we lack such a language suggested by Woodger and others throughout the course of the 20th century as we have continued to rely on natural language heuristics to communicate and reason about biological systems.

Category theory \cite{Lane1985,Lane1998,MacLane1992,Lawvere1997,Lawvere2003,Awodey2006} is a language that has been suggested since Woodger to provide a framework for representing and reasoning about biological systems \cite{GOGUEN1979,Ehresmann2007,Louie2009}. What is immediately useful about this language from the perspective of biology and its chemistry is that it presents as primitive the notion of transformation or interaction between objects. In fact, from this point of view, a defining characteristic of any entity (e.g. a protein, cell, organism, or population) is the set of relationships between it and other entities under consideration. Intrinsic properties are taken into account implicitly in this framework since the possible set of relationships between any object and any other is constrained by the nature of its intrinsic properties. 

What may be less obvious at first meeting is the way in which concepts from category theory, especially with regard to its interface with logic and geometry \cite{MacLane1992,Jacobs1998}, enable the formulation of what might be viewed as a framework for explaining the nature and development of so-called \emph{emergent properties} of systems that are able to represent information in molecular terms to take an arbitrary base level, but build upon this to generate the capacity to process information at higher levels of organization as well. We note that the existence of {\it levels of organization} that may have come into existence during the evolutionary process thus far, although common in various informal taxonomic systems used in biology, is a hypothesis that we hope to render verifiable in the long term through the introduction of a language capable of making predictions that could help distinguish the case in which there is from that in which there is not justification to reify the conceptual levels of organization that have been introduced to organize biological knowledge. The concept of separation of spatio-temporal scales, which is used to justify the consideration of one level of organization at a time in the biological model-building process \cite{Gunawardena2012,Karr2012}, may make it impossible to explain evolutionary processes that transcend such levels of organization providing one justification to our search for a language that can support model-building that can explain how biological systems might integrate information across levels of organization. The expression discussed here is general enough that it can be equally well applied to the consideration of relationships between any levels of biological organization assuming that any at all exist. Of course, it will need to be further specialized to be of use in specific contexts. However, the unity deriving from the judicious definition of underlying categorical concepts along the way is precisely the type of abstraction that we argue is necessary to enable conceptual compression of biological information and thereby improve our capacity to communicate and synthesize existing and future knowledge about biological systems.

Here we explain the concepts from category theory necessary to understand the way in which interacting objects at one level of organization (e.g. molecules) can produce phenomena that would themselves be identifiable as derived objects (e.g. cells) that justify the very conceptualization of a \emph{level of organization} in the first instance. We focus exclusively on defining the boundary conditions relating levels (e.g. molecules and cells or cells and multicellular organisms) of organization, which are necessary to understand in the course of defining a dynamical system that could model the \emph{evolution} of such. What results is a refinement of the relationship among levels of organization that are ubiquitously employed in biology and examples of which are used so far only heuristically as guides to pre-existing intuition (e.g. in the hierarchical nature of taxonomy and in the characterization of transcriptional and epigenetic among other levels of regulation of gene expression). Understanding how information is integrated via biological processes across such levels of organization is fundamental to the understanding of so-called complex phenotypes and the associated set of contingencies necessary to account for in proposed methods targeted at controlling or otherwise manipulating them.

\section{Conclusion}

The goal of organizing knowledge about biological systems will ultimately require the development of languages that can be tailored specifically for that task. The purpose of developing such a language is to provide a precise framework for a qualitative phenomenology of biological systems such that, any piece of data that can be collected, can be integrated and used to deduce apparent consequences that in some sense attempt to optimize the production of new or classification of old hypotheses that may be distinguished on that basis. Despite the fact that abstract concepts from pure mathematics have had little explicit coupling to the study of biological systems, we find that certain concepts can be adapted to serve as a high-level nucleus for the development of such a domain-specific language for reasoning about biological systems. In order to realize such a goal, this language will need to be implemented in computational form rather than simply stated in abstract notation; however, clarification of the relationship between purely abstract concepts and concepts abstracted from the study of biological systems can help to make the costly choice of a framework to be implemented for the purposes of knowledge representation.

%------bibliography---%
\bibliographystyle{unsrt} 
\bibliography{bib/books,bib/papers}
%---------------------%


%-----------------------------------------------%
%                   appendix
%-----------------------------------------------%
\appendix
\section{Category theory}\label{app:CatTh}
\iftoggle{thmsty}{
\begin{definition}
\label{definition-category}
}{}
A {\it category} $\mathcal{C}$ is:
\begin{enumerate}
\item A set of objects $\Ob(\mathcal{C})$.
\item For each pair $x, y \in \Ob(\mathcal{C})$ a set of morphisms
$\Mor_\mathcal{C}(x, y)$.
\item For each triple $x, y, z\in \Ob(\mathcal{C})$ a composition
map $ \Mor_\mathcal{C}(y, z) \times \Mor_\mathcal{C}(x, y)
\to \Mor_\mathcal{C}(x, z) $, denoted $(\phi, \psi) \mapsto
\phi \circ \psi$.
\end{enumerate}
Such that these constraints are satisfied:
\begin{enumerate}
\item For every element $x\in \Ob(\mathcal{C})$ there exists a
morphism $\text{id}_x\in \Mor_\mathcal{C}(x, x)$ such that
$\text{id}_x \circ \phi = \phi$ and $\psi \circ \text{id}_x = \psi $.
\item Composition is associative, i.e., $(\phi \circ \psi) \circ \chi =
\phi \circ ( \psi \circ \chi)$.
\end{enumerate}
\iftoggle{thmsty}{
\end{definition}
}

\iftoggle{thmsty}{
\begin{definition}
\label{definition-functor}
}{}
A {\it functor} $F : \mathcal{A} \to \mathcal{B}$
between two categories $\mathcal{A}, \mathcal{B}$ is:
\begin{enumerate}
\item A map $F : \Ob(\mathcal{A}) \to \Ob(\mathcal{B})$.
\item For every $x, y \in \Ob(\mathcal{A})$ a map
$F : \Mor_\mathcal{A}(x, y) \to \Mor_\mathcal{B}(F(x), F(y))$,
denoted $\phi \mapsto F(\phi)$.
\end{enumerate}
These data should be compatible with composition and identity morphisms
in the following manner: $F(\phi \circ \psi) =
F(\phi) \circ F(\psi)$ for a composable pair $(\phi, \psi)$ of
morphisms of $\mathcal{A}$ and $F(\text{id}_x) = \text{id}_{F(x)}$.
\iftoggle{thmsty}{
\end{definition}
}

\iftoggle{thmsty}{
\begin{definition}
\label{definition-transformation-functors}
}{}
Let $F, G : \mathcal{A} \to \mathcal{B}$ be functors.
A {\it natural transformation}, or a {\it morphism of functors}
$t : F \to G$, is a collection $\{t_x\}_{x\in \Ob(\mathcal{A})}$
such that
\begin{enumerate}
\item $t_x : F(x) \to G(x)$ is a morphism in the category $\mathcal{B}$, and
\item for every morphism $\phi : x \to y$ of $\mathcal{A}$ the following
diagram is commutative
$$
\xymatrix{
F(x) \ar[r]^{t_x} \ar[d]_{F(\phi)} & G(x) \ar[d]^{G(\phi)} \\
F(y) \ar[r]^{t_y} & G(y) }
$$
\end{enumerate}
\iftoggle{thmsty}{
\end{definition}
}

We can define a category having functors as objects and natural transformations as morphisms, which is called a functor category, by recognizing that every functor $F$ comes with the {\it identity} transformation $\text{id}_F : F \to F$. In addition, given a morphism of
functors $t : F \to G$ and a morphism of functors $s : E \to F$
then the {\it composition} $t \circ s$ is defined by the rule
$$
(t \circ s)_x = t_x \circ s_x : E(x) \to G(x)
$$
for $x \in \Ob(\mathcal{A})$.
This is a morphism of functors
from $E$ to $G$.
Thus, given categories
$\mathcal{A}$ and $\mathcal{B}$ we obtain the category of functors between $\mathcal{A}$ and
$\mathcal{B}$.

\iftoggle{thmsty}{
\begin{definition}
\label{definition-equivalence-categories}
}{}
An {\it equivalence of categories}
$F : \mathcal{A} \to \mathcal{B}$ is a functor such that there
exists a functor $G : \mathcal{B} \to \mathcal{A}$ such that
the compositions $F \circ G$ and $G \circ F$ are isomorphic to the
identity functors $\text{id}_\mathcal{B}$,
respectively $\text{id}_\mathcal{A}$.
In this case we say that $G$ is a {\it quasi-inverse} to $F$.
\iftoggle{thmsty}{
\end{definition}
}

\iftoggle{thmsty}{
\begin{definition}
\label{definition-adjoint}
}{}
Let $\mathcal{C}$, $\mathcal{D}$ be categories.
Let $u : \mathcal{C} \to \mathcal{D}$ and
$v : \mathcal{D} \to \mathcal{C}$ be functors.
We say that $u$ is a {\it left adjoint} of $v$ or that
$v$ is a {\it right adjoint} to $u$, written $u \dashv v$, if there are bijections
$$
\phi_{X,Y}:\Mor_\mathcal{D}(u(X), Y)
\simeq
\Mor_\mathcal{C}(X, v(Y))
$$
functorial in $X \in \Ob(\mathcal{C})$, and
$Y \in \Ob(\mathcal{D})$.
\iftoggle{thmsty}{
\end{definition}
}

Morphisms that are associated with each other according to the bijections of an adjunction are called {\it adjoint transposes} of one another. If $g:u(X) \rightarrow Y$, $g \in \Mor(\cD)$ then $g^*: X \rightarrow v(Y)$, $g^* \in \Mor(\cC)$ is given by $\phi_{X,Y}(g) = g^*$. Similarly for $f: X \rightarrow v(Y)$, $f \in \Mor(\cC)$ with $f^*:u(X) \rightarrow Y$, $f^* \in \Mor(\cD)$ is given by $\phi_{X,Y}^{-1}(f) = f^*$. We see then that $g^* = f$ and $f^* = g$.

\iftoggle{thmsty}{
\begin{definition}
\label{definition-opposite}
}{}
Given a category $\mathcal{C}$ the {\it opposite category}
$\mathcal{C}^{opp}$ is the category with the same objects
as $\mathcal{C}$ but all morphisms reversed.
\iftoggle{thmsty}{
\end{definition}
}

\iftoggle{thmsty}{
\begin{definition}
\label{definition-contravariant}
}{}
Let $\mathcal{C}$, $\mathcal{S}$ be categories.
A {\it contravariant} functor $F$
from $\mathcal{C}$ to $\mathcal{S}$
is a functor $\mathcal{C}^{opp}\to \mathcal{S}$.
\iftoggle{thmsty}{
\end{definition}
}

\iftoggle{thmsty}{
\begin{definition}
\label{definition-presheaf}
}{}
Let $\mathcal{C}$ be a category.
\begin{enumerate}
\item A {\it presheaf of sets on $\mathcal{C}$}
or simply a {\it presheaf} is a contravariant functor
$F$ from $\mathcal{C}$ to $\textit{Sets}$. When $F$ is a covariant functor $F : \mathcal{C}^{opp} \rightarrow \textit{Sets}$.
\item The category of presheaves is denoted $\textit{PSh}(\mathcal{C})$.
\end{enumerate}
\iftoggle{thmsty}{
\end{definition}
}

\iftoggle{thmsty}{
\begin{definition}
\label{definition-products}
}{}

Let $x, y\in \Ob(\mathcal{C})$,
A {\it product} of $x$ and $y$ is
an object $x \times y \in \Ob(\mathcal{C})$
together with morphisms
$p\in \Mor_{\mathcal C}(x \times y, x)$ and
$q\in\Mor_{\mathcal C}(x \times y, y)$ such
that the following universal property holds: for
any $w\in \Ob(\mathcal{C})$ and morphisms
$\alpha \in \Mor_{\mathcal C}(w, x)$ and
$\beta \in \Mor_\mathcal{C}(w, y)$
there is a unique
$\gamma\in \Mor_{\mathcal C}(w, x \times y)$ making
the diagram
$$
\xymatrix{
w \ar[rrrd]^\beta \ar@{-->}[rrd]_\gamma \ar[rrdd]_\alpha & & \\
& & x \times y \ar[d]_p \ar[r]_q & z \\
& & x &
}
$$
commute.
\iftoggle{thmsty}{
\end{definition}
}

\iftoggle{thmsty}{
\begin{definition}
\label{definition-has-products-of-pairs}
}{}
We say the category $\mathcal{C}$ {\it has products of pairs
of objects} if a product $x \times y$
exists for any $x, y \in \Ob(\mathcal{C})$.
\iftoggle{thmsty}{
\end{definition}
}

\iftoggle{thmsty}{
\begin{definition}
\label{definition-coproducts}
}{}
Let $x, y \in \Ob(\mathcal{C})$,
A {\it coproduct}, or {\it amalgamated sum} of $x$ and $y$ is
an object $x \amalg y \in \Ob(\mathcal{C})$
together with morphisms
$i \in \Mor_{\mathcal C}(x, x \amalg y)$ and
$j \in \Mor_{\mathcal C}(y, x \amalg y)$ such
that the following universal property holds: for
any $w \in \Ob(\mathcal{C})$ and morphisms
$\alpha \in \Mor_{\mathcal C}(x, w)$ and
$\beta \in \Mor_\mathcal{C}(y, w)$
there is a unique
$\gamma \in \Mor_{\mathcal C}(x \amalg y, w)$ making
the diagram
$$
\xymatrix{
& y \ar[d]^j \ar[rrdd]^\beta \\
x \ar[r]^i \ar[rrrd]_\alpha & x \amalg y \ar@{-->}[rrd]^\gamma \\
& & & w
}
$$
commute.
\iftoggle{thmsty}{
\end{definition}
}

\iftoggle{thmsty}{
\begin{definition}
\label{definition-has-coproducts-of-pairs}
}{}
We say the category $\mathcal{C}$ {\it has coproducts of pairs
of objects} if a coproduct $x \amalg y$
exists for any $x, y \in \Ob(\mathcal{C})$.
\iftoggle{thmsty}{
\end{definition}
}

\iftoggle{thmsty}{
\begin{definition}
\label{definition-product-category}
}{}
Let $\mathcal{A}$, $\mathcal{B}$ be categories.
The {\it product category} is the category
$\mathcal{A} \times \mathcal{B}$ with
objects
$\Ob(\mathcal{A} \times \mathcal{B}) =
\Ob(\mathcal{A}) \times \Ob(\mathcal{B})$
and
$$
\Mor_{\mathcal{A} \times \mathcal{B}}((x, y), (x', y'))
:=
\Mor_\mathcal{A}(x, x')\times
\Mor_\mathcal{B}(y, y').
$$
Composition of morphisms is defined according to components.
\iftoggle{thmsty}{
\end{definition}
}

\iftoggle{thmsty}{
\begin{definition}
\label{definition-bifunctor}
}{}
Given categories $\mathcal{C}_1$, $\mathcal{C}_2$, and $\mathcal{D}$. A {\it bifunctor} or binary functor or 2-ary functor or functor of two variables, $F$, is a functor whose domain is the product of two categories $F: \mathcal{C}_1 \times \mathcal{C}_2 \rightarrow \mathcal{D}$.
\iftoggle{thmsty}{
\end{definition}
}

\iftoggle{thmsty}{
\begin{definition}
\label{definition-hom-functor}
}{}
The {\it hom-functor} is a bifunctor defined on the product of a category $\mathcal{C}$ with its self-dual category $\mathcal{C}^{opp}$, which takes values in the category $\textit{Sets}$. Thus for a category $C$ its hom-functor is 
$$
hom(-,-): \mathcal{C}^{opp} \times \mathcal{C} \rightarrow \textit{Sets},
$$
which can be curried in two ways
\begin{eqnarray*}
hom^{(-)} &:& \mathcal{C}^{opp} \rightarrow \textit{Sets}^{\mathcal{C}},\\
hom_{(-)} &:& \mathcal{C} \rightarrow \textit{Sets}^{\mathcal{C}^{opp}}.
\end{eqnarray*}
The hom-functor maps
\begin{enumerate}
\item objects $(c,c') \in \mathcal{C}^{opp} \times \mathcal{C}$ to the hom-set $\Mor_{\mathcal{C}} (c,c')$, which is the set of morphisms in $\mathcal{C}$ with domain $c$ and codomain $c'$.
\item morphisms 
$$
(f,g):(c,c') \rightarrow (d,d') \in \Mor(\mathcal{C}^{opp} \times \mathcal{C}),
$$
where $f:d \rightarrow c \in \Mor(\mathcal{C})$ and $g:c' \rightarrow d' \in \Mor(\mathcal{C})$, to the set function
\begin{eqnarray*}
\Mor_{\mathcal{C}}(c,c') &\rightarrow& \Mor_{\mathcal{C}}(d,d')\\
(c \rightarrow c') &\mapsto& (d \rightarrow c \rightarrow c' \rightarrow d')
\end{eqnarray*}
\end{enumerate}
\iftoggle{thmsty}{
\end{definition}
}

\iftoggle{thmsty}{
\begin{definition}
\label{definition-representable-functor}
}{}
For a hom-functor $hom(-,-): \mathcal{C}^{opp} \times \mathcal{C} \rightarrow \textit{Sets}$ for $c \in \Ob(\mathcal{C})$ a covariant and contravariant functor can be derived by specializing the hom-functor to morphisms out of or into the object $c$ as
\begin{eqnarray*}
h^c \equiv hom(-,c) &:& \mathcal{C}^{opp} \rightarrow \textit{Sets}\\
h_c \equiv hom(c,-) &:& \mathcal{C} \rightarrow \textit{Sets}.
\end{eqnarray*}
Functors that are isomorphic to $h^c$ or $h_c$ are referred to as {\it corepresentable or representable functors} respectively and $c$ is their {\it representing object}. Note that $h^c \in \Ob(\textit{PSh}(\mathcal{C}))$.
\iftoggle{thmsty}{
\end{definition}
}{}

\begin{figure}
\noindent\includegraphics[width=0.7\columnwidth]{fig/hom.pdf}
\caption{The presheaf represented by $c' \in \Ob(\cC)$ is $h^{c'} : \cC^{opp} \rightarrow \textit{Sets}$. It sends objects to the set of morphisms in which they are the domain object with codomain $c'$ and morphisms $f:d \rightarrow c \in \Mor{\cC}$ to set functions $h^{c'} \circ f: \Mor_{\cC}(c,c') \rightarrow \Mor_{\cC}(d,c')$ via pre-composition.}
\label{fig:hom}
\end{figure}

The action of $h^c$ on objects and morphisms is summarized in figure \ref{fig:hom}. The preceding definitions are standard category theoretic constructions. Ellerman has proposed an interpretation of adjoint functors, that demonstrates their relevance to the concept of information encoding and decoding (or sending and receiving) in the context of biological systems \cite{Ellerman2005}.

\iftoggle{thmsty}{
\begin{definition}
\label{definition-birepresentable}
}{}
A bifunctor $bif (-,-): \cC^{opp} \times \cD \rightarrow \textit{Sets}$ is said to be {\it birepresentable} if there exists a pair of functors $F:\cC^{opp} \rightleftarrows \cD:G$ where $c \in \Ob(\cC^{opp})$ and $d \in \Ob(\cD)$ gives
\begin{eqnarray*}
b^{d} \equiv bif(-,d) &:& \mathcal{C}^{opp} \rightarrow \textit{Sets},\\
b_{c} \equiv bif(c,-) &:& \mathcal{D} \rightarrow \textit{Sets}.
\end{eqnarray*}
natural in $c$ and $d$ such that $F \dashv G$. The functors $b^{d}$ and $b_{c}$ are defined on
\begin{enumerate}
\item objects for all $c_i \in \Ob(\cC^{opp})$ and for all $d_i \in \Ob(\cD)$
\begin{eqnarray*}
b^{d} (c_i) &=& \Mor_{\cC^{opp}}(c_i,Gd),\\
b_{c} (d_i) &=& \Mor_{\cD}(Fc,d_i).
\end{eqnarray*}

\item morphisms for all $f_{ij}:c_j \rightarrow c_i \in \Mor(\cC^{opp})$ and for all $g_{ij}:d_i \rightarrow d_j \in \Mor(\cD)$ as
\begin{eqnarray*}
b^{d} (f_{ij}) &:& \Mor_{\cC^{opp}}(c_i,Gd) \rightarrow \Mor_{\cC^{opp}}(c_j,Gd),\\
b_{c} (g_{ij}) &:& \Mor_{\cD}(Fc,d_i) \rightarrow \Mor_{\cD}(Fc,d_j).
\end{eqnarray*}
\end{enumerate}
\iftoggle{thmsty}{
\end{definition}
}

\section{Sieves and sheaves}\label{Sheaves}

\iftoggle{thmsty}{
\begin{definition}
\label{definition-presheaves-injective-surjective}
}{}
Let $\mathcal{C}$ be a category, and let $\varphi : \mathcal{F}
\to \mathcal{G}$ be a map of presheaves of sets.
\begin{enumerate}
\item We say that $\varphi$ is {\it injective} if for every object
$U$ of $\mathcal{C}$ we have $\alpha : \mathcal{F}(U)
\to \mathcal{G}(U)$ is injective.
\item We say that $\varphi$ is {\it surjective} if for every object
$U$ of $\mathcal{C}$ we have $\alpha : \mathcal{F}(U)
\to \mathcal{G}(U)$ is surjective.
\end{enumerate}
\iftoggle{thmsty}{
\end{definition}
}

\iftoggle{thmsty}{
\begin{lemma}
\label{lemma-mono-epi}
}{}
The injective (resp.\ surjective) maps defined above
are exactly the monomorphisms (resp.\ epimorphisms) of
$\textit{PSh}(\mathcal{C})$. A map is an isomorphism
if and only if it is both injective and surjective.
\iftoggle{thmsty}{
\end{lemma}
}{}

\iftoggle{thmsty}{
\begin{definition}
\label{definition-sub-presheaf}
}{}
We say $\mathcal{F}$ is a {\it subpresheaf} of $\mathcal{G}$
if for every object $U \in \Ob(\mathcal{C})$ the set
$\mathcal{F}(U)$ is a subset of $\mathcal{G}(U)$, compatibly
with the restriction mappings.
\iftoggle{thmsty}{
\end{definition}
}

\noindent
In other words, the inclusion
maps $\mathcal{F}(U) \to \mathcal{G}(U)$
glue together to give an (injective) morphism of
presheaves $\mathcal{F} \to \mathcal{G}$.

\iftoggle{thmsty}{
\begin{lemma}
\label{lemma-image}
}{}
Let $\mathcal{C}$ be a category.
Suppose that $\varphi : \mathcal{F} \to \mathcal{G}$ is a
morphism of presheaves of sets on $\mathcal{C}$.
There exists a unique subpresheaf $\mathcal{G}' \subset \mathcal{G}$
such that $\varphi$ factors as
$\mathcal{F} \to \mathcal{G}' \to \mathcal{G}$
and such that the first map is surjective. We
say that $\mathcal{G}'$ is the {\it image of $\varphi$}.
\iftoggle{thmsty}{
\end{lemma}
}

\iftoggle{thmsty}{
\begin{definition}
\label{definition-sieve-s}
}{}
Let $\mathcal{C}$ be a category. Let $U \in \Ob(\mathcal{C})$.
A {\it sieve $S$ on $U$} is a subpresheaf $S \subset h_U$.
\iftoggle{thmsty}{
\end{definition}
}

\noindent
In other words, a sieve on $U$ picks out for each object
$T \in \Ob(\mathcal{C})$ a subset $S(T)$ of the set
of all morphisms $T \to U$. In fact, the only condition
on the collection of subsets
$S(T) \subset h_U(T) = \Mor_\mathcal{C}(T, U)$
is the following rule
\begin{equation}
\label{equation-property-sieve}
\left.
\begin{matrix}
(\alpha : T \to U) \in S(T) \\
g : T' \to T
\end{matrix}
\right\} \Rightarrow
(\alpha \circ g : T' \to U) \in S(T')
\end{equation}

\iftoggle{thmsty}{
\begin{lemma}
\label{lemma-sieves-set}
}{}
Let $\mathcal{C}$ be a category. Let $U \in \Ob(\mathcal{C})$.
\begin{enumerate}
\item The collection of sieves on $U$ is a set.
\item Inclusion defines a partial ordering on this set.
\item Unions and intersections of sieves are sieves.
\item
\label{item-sieve-generated}
Given a family of morphisms $\{U_i \to U\}_{i\in I}$
of $\mathcal{C}$ with target $U$
there exists a unique smallest sieve $S$ on $U$ such that
each $U_i \to U$ belongs to $S(U_i)$.
\item The sieve $S = h_U$ is the maximal sieve.
\item The empty subpresheaf is the minimal sieve.
\end{enumerate}
\iftoggle{thmsty}{
\end{lemma}
}

\iftoggle{thmsty}{
\begin{proof}
}{}
By our definition of subpresheaf, the collection of
all subpresheaves of a presheaf $\mathcal{F}$ is a subset of
$\prod_{U \in \Ob(\mathcal{C})} \mathcal{P}(\mathcal{F}(U))$.
And this is a set. (Here $\mathcal{P}(A)$ denotes
the powerset of $A$.) Hence the collection of sieves on $U$
is a set.

\medskip\noindent
The partial ordering is defined by: $S \leq S'$ if and only if
$S(T) \subset S'(T)$ for all $T \to U$. Notation: $S \subset S'$.

\medskip\noindent
Given a collection of sieves $S_i$, $i \in I$ on $U$ we can
define $\bigcup S_i$ as the sieve with values
$(\bigcup S_i)(T) = \bigcup S_i(T)$ for all
$T \in \Ob(\mathcal{C})$.
We define the intersection $\bigcap S_i$ in the same way.

\medskip\noindent
Given $\{U_i \to U\}_{i\in I}$ as in the statement, consider
the morphisms of presheaves $h_{U_i} \to h_U$. We simply
define $S$ as the union of the images of these maps of presheaves.
\iftoggle{thmsty}{
\end{proof}
}

\iftoggle{thmsty}{
\begin{definition}
\label{definition-sieve-generated}
}{}
Let $\mathcal{C}$ be a category.
Given a family of morphisms $\{f_i : U_i \to U\}_{i\in I}$
of $\mathcal{C}$ with target $U$ we say the sieve
$S$ on $U$ is the {\it sieve  on $U$
generated by the morphisms $f_i$}.
\iftoggle{thmsty}{
\end{definition}
}


\iftoggle{thmsty}{
\begin{definition}
\label{definition-pullback-sieve}
}{}
Let $\mathcal{C}$ be a category.
Let $f : V \to U$ be a morphism of $\mathcal{C}$.
Let $S \subset h_U$ be a sieve. We define the
{\it pullback of $S$ by $f$} to be the sieve
$S \times_U V$ of $V$ defined by the rule
$$
(\alpha : T \to V) \in (S \times_U V)(T)
\Leftrightarrow
(f \circ \alpha : T \to U) \in S(T)
$$
\iftoggle{thmsty}{
\end{definition}
}

$S \times_U V$ can also be referred to as the {\it base change}
of $S$ by $f : V \to U$.

\iftoggle{thmsty}{
\begin{lemma}
\label{lemma-pullback-sieve-section}
}{}
Let $\mathcal{C}$ be a category.
Let $U \in \Ob(\mathcal{C})$.
Let $S$ be a sieve on $U$.
If $f : V \to U$ is in $S$, then
$S \times_U V = h_V$ is maximal.
\iftoggle{thmsty}{
\end{lemma}
}

\iftoggle{thmsty}{
\begin{definition}
\label{definition-topology}
}{}
Let $\mathcal{C}$ be a category.
A {\it topology on $\mathcal{C}$} is given by the following
datum:
\begin{list}{}{}
\item For every $U \in \Ob(\mathcal{C})$
a subset $J(U)$ of the set of all sieves on $U$.
\end{list}
These sets $J(U)$ have to satisfy the following
conditions
\begin{enumerate}
\item For every morphism $f : V \to U$ in $\mathcal{C}$, and
every element $S \in J(U)$ the pullback $S \times_U V$
is an element of $J(V)$.
\item If $S$ and $S'$ are sieves on $U \in \Ob(\mathcal{C})$,
if $S \in J(U)$, and if for all $f \in S(V)$ the pullback
$S' \times_U V$ belongs to $J(V)$, then $S'$ belongs to $J(U)$.
\item For every $U \in \Ob(\mathcal{C})$ the
maximal sieve $S = h_U$ belongs to $J(U)$.
\end{enumerate}
\iftoggle{thmsty}{
\end{definition}
}

\noindent
In this case, the sieves belonging to $J(U)$ are called
the {\it covering sieves}. 

\iftoggle{thmsty}{
\begin{definition}
\label{definition-family-morphisms-fixed-target}
}{}
Let $\mathcal{C}$ be a category.
A {\it family of morphisms with fixed target} in $\mathcal{C}$ is
given by an object $U \in \Ob(\mathcal{C})$, a set $I$ and
for each $i\in I$ a morphism $U_i \to U$ of $\mathcal{C}$ with target $U$.
We use the notation $\{U_i \to U\}_{i\in I}$ to indicate this.
\iftoggle{thmsty}{
\end{definition}
}

\noindent This
notation is meant to suggest an open covering as in topology.

\iftoggle{thmsty}{
\begin{definition}
\label{definition-site-s}
}{}
A {\it site} is given by a category $\mathcal{C}$ and a set
$\text{Cov}(\mathcal{C})$ of families of morphisms with fixed target
$\{U_i \to U\}_{i \in I}$, called {\it coverings of $\mathcal{C}$},
satisfying the following axioms
\begin{enumerate}
\item If $V \to U$ is an isomorphism then $\{V \to U\} \in
\text{Cov}(\mathcal{C})$.
\item If $\{U_i \to U\}_{i\in I} \in \text{Cov}(\mathcal{C})$ and for each
$i$ we have $\{V_{ij} \to U_i\}_{j\in J_i} \in \text{Cov}(\mathcal{C})$, then
$\{V_{ij} \to U\}_{i \in I, j\in J_i} \in \text{Cov}(\mathcal{C})$.
\item If $\{U_i \to U\}_{i\in I}\in \text{Cov}(\mathcal{C})$
and $V \to U$ is a morphism of $\mathcal{C}$ then $U_i \times_U V$
exists for all $i$ and
$\{U_i \times_U V \to V \}_{i\in I} \in \text{Cov}(\mathcal{C})$.
\end{enumerate}
\iftoggle{thmsty}{
\end{definition}
}

We can now define what a sheaf is in two different ways.

\iftoggle{thmsty}{
\begin{definition}
\label{definition-sheaf-sets}
}{}
Let $\mathcal{C}$ be a site, and let $\mathcal{F}$ be a presheaf of sets
on $\mathcal{C}$. We say $\mathcal{F}$ is a {\it sheaf} if
for every covering $\{U_i \to U\}_{i \in I} \in \text{Cov}(\mathcal{C})$
the diagram
\begin{equation}
\label{equation-sheaf-condition}
\xymatrix{
\mathcal{F}(U) \ar[r]
&
\prod\nolimits_{i\in I}
\mathcal{F}(U_i)
\ar@<1ex>[r]^-{\text{pr}_0^*} \ar@<-1ex>[r]_-{\text{pr}_1^*}
&
\prod\nolimits_{(i_0, i_1) \in I \times I}
\mathcal{F}(U_{i_0} \times_U U_{i_1})
}
\end{equation}
represents the first arrow as the equalizer of $\text{pr}_0^*$
and $\text{pr}_1^*$.
\iftoggle{thmsty}{
\end{definition}
}

\iftoggle{thmsty}{
\begin{definition}
\label{definition-sheaf-sets-topology}
}{}
Let $\mathcal{C}$ be a category endowed with a
topology $J$. Let $\mathcal{F}$ be a presheaf of sets
on $\mathcal{C}$.
We say that $\mathcal{F}$ is a
{\it sheaf} on $\mathcal{C}$
if for every $U \in \Ob(\mathcal{C})$ and for
every covering sieve $S$ of $U$ the canonical map
$$
\Mor_{\textit{PSh}(\mathcal{C})}(h_U, \mathcal{F})
\longrightarrow
\Mor_{\textit{PSh}(\mathcal{C})}(S, \mathcal{F})
$$
is bijective.
\iftoggle{thmsty}{
\end{definition}
}

\noindent
The left hand side of the formula equals $\mathcal{F}(U)$ according to the Yoneda lemma. In other words, $\mathcal{F}$ is a sheaf if and only if a section of $\mathcal{F}$
over $U$ is the same thing as a compatible collection of sections
$s_{T, \alpha} \in \mathcal{F}(T)$ parametrized by $(\alpha : T \to U) \in S(T)$, and this for every covering sieve $S$ on $U$.

\end{document}