\iftoggle{thmsty}{
\begin{definition}
\label{definition-presheaves-injective-surjective}
}{}
Let $\mathcal{C}$ be a category, and let $\varphi : \mathcal{F}
\to \mathcal{G}$ be a map of presheaves of sets.
\begin{enumerate}
\item We say that $\varphi$ is {\it injective} if for every object
$U$ of $\mathcal{C}$ we have $\alpha : \mathcal{F}(U)
\to \mathcal{G}(U)$ is injective.
\item We say that $\varphi$ is {\it surjective} if for every object
$U$ of $\mathcal{C}$ we have $\alpha : \mathcal{F}(U)
\to \mathcal{G}(U)$ is surjective.
\end{enumerate}
\iftoggle{thmsty}{
\end{definition}
}

\iftoggle{thmsty}{
\begin{lemma}
\label{lemma-mono-epi}
}{}
The injective (resp.\ surjective) maps defined above
are exactly the monomorphisms (resp.\ epimorphisms) of
$\textit{PSh}(\mathcal{C})$. A map is an isomorphism
if and only if it is both injective and surjective.
\iftoggle{thmsty}{
\end{lemma}
}{}

\iftoggle{thmsty}{
\begin{definition}
\label{definition-sub-presheaf}
}{}
We say $\mathcal{F}$ is a {\it subpresheaf} of $\mathcal{G}$
if for every object $U \in \Ob(\mathcal{C})$ the set
$\mathcal{F}(U)$ is a subset of $\mathcal{G}(U)$, compatibly
with the restriction mappings.
\iftoggle{thmsty}{
\end{definition}
}

\noindent
In other words, the inclusion
maps $\mathcal{F}(U) \to \mathcal{G}(U)$
glue together to give an (injective) morphism of
presheaves $\mathcal{F} \to \mathcal{G}$.

\iftoggle{thmsty}{
\begin{lemma}
\label{lemma-image}
}{}
Let $\mathcal{C}$ be a category.
Suppose that $\varphi : \mathcal{F} \to \mathcal{G}$ is a
morphism of presheaves of sets on $\mathcal{C}$.
There exists a unique subpresheaf $\mathcal{G}' \subset \mathcal{G}$
such that $\varphi$ factors as
$\mathcal{F} \to \mathcal{G}' \to \mathcal{G}$
and such that the first map is surjective. We
say that $\mathcal{G}'$ is the {\it image of $\varphi$}.
\iftoggle{thmsty}{
\end{lemma}
}

\iftoggle{thmsty}{
\begin{definition}
\label{definition-sieve-s}
}{}
Let $\mathcal{C}$ be a category. Let $U \in \Ob(\mathcal{C})$.
A {\it sieve $S$ on $U$} is a subpresheaf $S \subset h_U$.
\iftoggle{thmsty}{
\end{definition}
}

\noindent
In other words, a sieve on $U$ picks out for each object
$T \in \Ob(\mathcal{C})$ a subset $S(T)$ of the set
of all morphisms $T \to U$. In fact, the only condition
on the collection of subsets
$S(T) \subset h_U(T) = \Mor_\mathcal{C}(T, U)$
is the following rule
\begin{equation}
\label{equation-property-sieve}
\left.
\begin{matrix}
(\alpha : T \to U) \in S(T) \\
g : T' \to T
\end{matrix}
\right\} \Rightarrow
(\alpha \circ g : T' \to U) \in S(T')
\end{equation}

\iftoggle{thmsty}{
\begin{lemma}
\label{lemma-sieves-set}
}{}
Let $\mathcal{C}$ be a category. Let $U \in \Ob(\mathcal{C})$.
\begin{enumerate}
\item The collection of sieves on $U$ is a set.
\item Inclusion defines a partial ordering on this set.
\item Unions and intersections of sieves are sieves.
\item
\label{item-sieve-generated}
Given a family of morphisms $\{U_i \to U\}_{i\in I}$
of $\mathcal{C}$ with target $U$
there exists a unique smallest sieve $S$ on $U$ such that
each $U_i \to U$ belongs to $S(U_i)$.
\item The sieve $S = h_U$ is the maximal sieve.
\item The empty subpresheaf is the minimal sieve.
\end{enumerate}
\iftoggle{thmsty}{
\end{lemma}
}

\iftoggle{thmsty}{
\begin{proof}
}{}
By our definition of subpresheaf, the collection of
all subpresheaves of a presheaf $\mathcal{F}$ is a subset of
$\prod_{U \in \Ob(\mathcal{C})} \mathcal{P}(\mathcal{F}(U))$.
And this is a set. (Here $\mathcal{P}(A)$ denotes
the powerset of $A$.) Hence the collection of sieves on $U$
is a set.

\medskip\noindent
The partial ordering is defined by: $S \leq S'$ if and only if
$S(T) \subset S'(T)$ for all $T \to U$. Notation: $S \subset S'$.

\medskip\noindent
Given a collection of sieves $S_i$, $i \in I$ on $U$ we can
define $\bigcup S_i$ as the sieve with values
$(\bigcup S_i)(T) = \bigcup S_i(T)$ for all
$T \in \Ob(\mathcal{C})$.
We define the intersection $\bigcap S_i$ in the same way.

\medskip\noindent
Given $\{U_i \to U\}_{i\in I}$ as in the statement, consider
the morphisms of presheaves $h_{U_i} \to h_U$. We simply
define $S$ as the union of the images of these maps of presheaves.
\iftoggle{thmsty}{
\end{proof}
}

\iftoggle{thmsty}{
\begin{definition}
\label{definition-sieve-generated}
}{}
Let $\mathcal{C}$ be a category.
Given a family of morphisms $\{f_i : U_i \to U\}_{i\in I}$
of $\mathcal{C}$ with target $U$ we say the sieve
$S$ on $U$ is the {\it sieve  on $U$
generated by the morphisms $f_i$}.
\iftoggle{thmsty}{
\end{definition}
}


\iftoggle{thmsty}{
\begin{definition}
\label{definition-pullback-sieve}
}{}
Let $\mathcal{C}$ be a category.
Let $f : V \to U$ be a morphism of $\mathcal{C}$.
Let $S \subset h_U$ be a sieve. We define the
{\it pullback of $S$ by $f$} to be the sieve
$S \times_U V$ of $V$ defined by the rule
$$
(\alpha : T \to V) \in (S \times_U V)(T)
\Leftrightarrow
(f \circ \alpha : T \to U) \in S(T)
$$
\iftoggle{thmsty}{
\end{definition}
}

$S \times_U V$ can also be referred to as the {\it base change}
of $S$ by $f : V \to U$.

\iftoggle{thmsty}{
\begin{lemma}
\label{lemma-pullback-sieve-section}
}{}
Let $\mathcal{C}$ be a category.
Let $U \in \Ob(\mathcal{C})$.
Let $S$ be a sieve on $U$.
If $f : V \to U$ is in $S$, then
$S \times_U V = h_V$ is maximal.
\iftoggle{thmsty}{
\end{lemma}
}

\iftoggle{thmsty}{
\begin{definition}
\label{definition-topology}
}{}
Let $\mathcal{C}$ be a category.
A {\it topology on $\mathcal{C}$} is given by the following
datum:
\begin{list}{}{}
\item For every $U \in \Ob(\mathcal{C})$
a subset $J(U)$ of the set of all sieves on $U$.
\end{list}
These sets $J(U)$ have to satisfy the following
conditions
\begin{enumerate}
\item For every morphism $f : V \to U$ in $\mathcal{C}$, and
every element $S \in J(U)$ the pullback $S \times_U V$
is an element of $J(V)$.
\item If $S$ and $S'$ are sieves on $U \in \Ob(\mathcal{C})$,
if $S \in J(U)$, and if for all $f \in S(V)$ the pullback
$S' \times_U V$ belongs to $J(V)$, then $S'$ belongs to $J(U)$.
\item For every $U \in \Ob(\mathcal{C})$ the
maximal sieve $S = h_U$ belongs to $J(U)$.
\end{enumerate}
\iftoggle{thmsty}{
\end{definition}
}

\noindent
In this case, the sieves belonging to $J(U)$ are called
the {\it covering sieves}. 

\iftoggle{thmsty}{
\begin{definition}
\label{definition-family-morphisms-fixed-target}
}{}
Let $\mathcal{C}$ be a category.
A {\it family of morphisms with fixed target} in $\mathcal{C}$ is
given by an object $U \in \Ob(\mathcal{C})$, a set $I$ and
for each $i\in I$ a morphism $U_i \to U$ of $\mathcal{C}$ with target $U$.
We use the notation $\{U_i \to U\}_{i\in I}$ to indicate this.
\iftoggle{thmsty}{
\end{definition}
}

\noindent This
notation is meant to suggest an open covering as in topology.

\iftoggle{thmsty}{
\begin{definition}
\label{definition-site-s}
}{}
A {\it site} is given by a category $\mathcal{C}$ and a set
$\text{Cov}(\mathcal{C})$ of families of morphisms with fixed target
$\{U_i \to U\}_{i \in I}$, called {\it coverings of $\mathcal{C}$},
satisfying the following axioms
\begin{enumerate}
\item If $V \to U$ is an isomorphism then $\{V \to U\} \in
\text{Cov}(\mathcal{C})$.
\item If $\{U_i \to U\}_{i\in I} \in \text{Cov}(\mathcal{C})$ and for each
$i$ we have $\{V_{ij} \to U_i\}_{j\in J_i} \in \text{Cov}(\mathcal{C})$, then
$\{V_{ij} \to U\}_{i \in I, j\in J_i} \in \text{Cov}(\mathcal{C})$.
\item If $\{U_i \to U\}_{i\in I}\in \text{Cov}(\mathcal{C})$
and $V \to U$ is a morphism of $\mathcal{C}$ then $U_i \times_U V$
exists for all $i$ and
$\{U_i \times_U V \to V \}_{i\in I} \in \text{Cov}(\mathcal{C})$.
\end{enumerate}
\iftoggle{thmsty}{
\end{definition}
}

We can now define what a sheaf is in two different ways.

\iftoggle{thmsty}{
\begin{definition}
\label{definition-sheaf-sets}
}{}
Let $\mathcal{C}$ be a site, and let $\mathcal{F}$ be a presheaf of sets
on $\mathcal{C}$. We say $\mathcal{F}$ is a {\it sheaf} if
for every covering $\{U_i \to U\}_{i \in I} \in \text{Cov}(\mathcal{C})$
the diagram
\begin{equation}
\label{equation-sheaf-condition}
\xymatrix{
\mathcal{F}(U) \ar[r]
&
\prod\nolimits_{i\in I}
\mathcal{F}(U_i)
\ar@<1ex>[r]^-{\text{pr}_0^*} \ar@<-1ex>[r]_-{\text{pr}_1^*}
&
\prod\nolimits_{(i_0, i_1) \in I \times I}
\mathcal{F}(U_{i_0} \times_U U_{i_1})
}
\end{equation}
represents the first arrow as the equalizer of $\text{pr}_0^*$
and $\text{pr}_1^*$.
\iftoggle{thmsty}{
\end{definition}
}

\iftoggle{thmsty}{
\begin{definition}
\label{definition-sheaf-sets-topology}
}{}
Let $\mathcal{C}$ be a category endowed with a
topology $J$. Let $\mathcal{F}$ be a presheaf of sets
on $\mathcal{C}$.
We say that $\mathcal{F}$ is a
{\it sheaf} on $\mathcal{C}$
if for every $U \in \Ob(\mathcal{C})$ and for
every covering sieve $S$ of $U$ the canonical map
$$
\Mor_{\textit{PSh}(\mathcal{C})}(h_U, \mathcal{F})
\longrightarrow
\Mor_{\textit{PSh}(\mathcal{C})}(S, \mathcal{F})
$$
is bijective.
\iftoggle{thmsty}{
\end{definition}
}

\noindent
The left hand side of the formula equals $\mathcal{F}(U)$ according to the Yoneda lemma. In other words, $\mathcal{F}$ is a sheaf if and only if a section of $\mathcal{F}$
over $U$ is the same thing as a compatible collection of sections
$s_{T, \alpha} \in \mathcal{F}(T)$ parametrized by $(\alpha : T \to U) \in S(T)$, and this for every covering sieve $S$ on $U$.
